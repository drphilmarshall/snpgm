\documentclass{article}[12pt]
\input{preamble}
\usepackage{graphicx}
\newcommand{\thetalctrue}{\theta^{lc-\rm{true}}}
\begin{document}
\section{Objectives}
The objectives for this document are to
\begin{itemize}
\item Have a description of the models, assumptions and detailed description of the mathematical formulae used. 
\item Explore similar, but slightly different models and methods to analyze them\end{itemize} 

\section{Model 1}
This is the model discussed in the first hack and described in Fig.~\ref{fig:model1} 
\begin{figure}[h]
\includegraphics{images/snpgm}
\caption{Model 1}
\label{fig:model1}
\end{figure}

\subsection{Observed Values}
There are three different quantities that are assumed to be observed values or 
`data' in the Bayesian sense.
\begin{itemize}
\item Cosmological redshift of each supernova, with $z_i$ denoting the redshift of the ith supernova.
\item Observed flux of i'th supernova in the j'th epoch with the associated bandpass, denoted by $f_{i,j}$.
\item The noise in each of the measurements of flux, which is due to noise in the number of photons from
the supernova-source, sky background as well as due to efficiency in image differencing. In this model, we will assume that this is totally due to the sky background which is measured independently, and perfectly. We will denote this by 
$\sigma^{sky}_{i,j}$ where, $i$ indexes the supernova, while $j$ indexes the epoch of observation of the supernova. 
\end{itemize}

\subsection{The Model}
This discussion is based on the `SALT2' model, with certain assumptions or 
interpretations beyond the exact model. 

The SALT2 model has two parts:
\begin{itemize}
    \item A light curve model with parameters $\{t_0, x_0, x_1, c, z\}.$ The model also has a likelihood for the flux values at different times:
        $P(f_{i,j} \vert \{t_0, x_0, x_1, c, z\}, \{\sigma_{i,j}\} )$
    \item A Tripp ansatz that relates the peak absolute magnitude $M_{b}$ in the rest frame BessellB band to the parameters ${x_1, c}$ and a global parameter $M$. 
    \be
    M_b = M + \beta c - \alpha x_1
    \ee
    so that the peak magnitude $m_B^{\star}$ in Bessell B band calculated from
    the model is related as
    \be
    \mu = m_B^{\star} - M_b = m_B^{\star} + \alpha x_1 - \beta c - M
    \ee
    where
    \be
    m_B^{\star} = -2.5 \log_{10}{(f_B^{\star})}, \quad f_B^{\star} = \int d\lambda T(\lambda) \frac{d S}{d\lambda}(\lambda, p=0)
    \ee
    which is related mostly to $x_0,$ since the contribution from terms 
    involving $c$ and $x_1$ to this quantity is constrained to be exactly $0$
    in defining the SALT2 model at $p=0$ at the central wavelength of the 
    BessellB band. In this very good approximation, $m_B^{\star} = -2.5 \log_{10}{(x_0)} -2.5 \log_{10}{(F_0)},$ where $F_0$ is a constant which can be evaluated for the SALT2 model. Thus, we will treat $x_0$ and $m_B^{\star}$ as equivalent quantities. 
\end{itemize}
An additional ingredient, we will be using is the interpretation of intrinsic
dispersion:
\begin{itemize}
 
    \item It is well known that standardization in the above manner is not complete, and the distance moduli derived using the above equations have residual scatter. There is some evidence that this scatter is different in different bands, and therefore cannot be completely modelled by a scatter in $M$ above, but we shall ignore this for now and model this by assuming that 
        \be M \sim N(0, \sigma_{int}) \ee  
     where $\sigma_{int}$ is the intrinsic dispersion which we will infer from 
     the study.
\end{itemize}
Combining these, we can obtain the probability distributions
$ P(m_B^{\star}\vert \Omega, M)$ or equivalently $P(x_0 \vert \Omega, M)$

\subsection{Posterior Samples of the Cosmology}
The quantity we are interested in is the posterior distribution of the cosmological parameters $\Omega$ (where we will use $\Omega$ to denote the set of
cosmological parameters, given the set of measurements $f_{i, j}$ , $\sigma_{i,j}$ which is 
\be
P(\Omega \vert \{f_{i,j}, \sigma_{i,j}, z_i,  \}) = 
        \int d\alpha d\beta P(\Omega, \alpha, \beta \vert \{f_{i,j}, \sigma_{i,j}, z_i\})
\ee

To get this quantity, we use:
\beqn
P(\Omega, \alpha, \beta \vert \{ f_{i,j}, \sigma_{i,j}, z_i\}) 
    &\propto& P(\{ f_{i,j}\} \vert \Omega, \alpha, \beta, \{\sigma_{i,j}, z_i \} ) P(\Omega, \alpha, \beta) 
    \\ 
    &=& \int  \Pi_{i} d\thetalctrue_{i} P(\{f_{i,j} \}, \vert \{\sigma_{i,j} , z_i, \thetalctrue_i, z_i \}) P(\{\thetalctrue_i \}\vert \Omega, \alpha, \beta) 
\eeqn

Where we have taken advantage of the fact that the fluxes $f_{i,j}$ of the supernovae only depend on the cosmological parameters $\Omega$ through the light 
curve parameters $\{\thetalctrue_i\}.$

Here, the set of parameters $\thetalctrue$ is defined by 
\be
    \thetalctrue = \{x_0, x_1, c, t_0, M \}
\ee
where we will assume that only $x_0$ is related to the cosmological parameters
$\Omega$ through the Tripp ansatz:
\be
\mu(\Omega, z ) = m_B^{\star} + \alpha x_1 - \beta c  - M 
\ee
$m_B^{\star}$ is the magnitude of the light curve model at peak in the $B$ band, and is approximately only related to the SN $x_0,$ as the constriants chosen in
the SALT model force contributions from the stretch term and color term to be zero at the central wavelength of the B band. We will take this relationship to 
be 
\be 
m_B^{\star} = - 2.5 \log_{10}(x_0)
\ee 
and absorb the effective width of the filter into M.
Here, $\mu(\Omega, z)$ is the distance modulus of the supernova calculated from the cosmological parameters and the redshift $z$ of the supernova. 

The model here is that supernovae have an 'standardized candle
Thus, values of $m_B^{\star},$ and consequently $x_0$ depend on the cosmological parameters, and also on the 
In this model, we are assuming that the cosmological redshift of each supernova $z_i$ is measured exactly. The other set of observed quantities are counts in 
each filter band for each supernova, denoted as $f_{i,j}$ the flux of the i'th 
supernova in the j'th band in physical units. 
As indicated in the diagrams, the flux of the i'th supernova in the j'th band is
determined by the parameters $\{x_{0,i}^{true}, x_{1,i}^{true}, c_{i}^{true}, t_0^{true}, z_i\},$ and a certain amount o
\end{document}
